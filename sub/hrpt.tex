
\part{Zukunft}
\section[]{High-resolution picture transmission(HRPT)}

Hrpt beschreibt den Nachfolger des APT Verfahrens. Es setzt im Vergleich zu diesem auf eine digitale Übertragung der Daten. Die Signale werden im L-Band auf Frequenzen um 1,7GHz ausgesendet. Die Übertragungen beschränken sich aber auf einfache Bilddaten in verschiedenen Wellenlängen des visuellen und infraroten Spektrums. Die Daten sind zu einem großen Teil die selben wie aus dem APT Verfahren aber mit einer deutlich erhöhten Qualität. Deshalb setzten neben Noaa auch Europäische Wetterorganisationen auch heute noch auf dieses Verfahren. Da die Antennen sehr klein werden wird eine Satellitenschüssel erforderlich um die Verluste der kleinen Antenne zu kompensieren. Hierzu eignet sich auch eine Helix Antenne die aufgrund der kleinen Größe einfacher zu bauen ist als für 137Mhz. Leider bewegt sich der Trend immer weiter weg von einfach zu empfangenden Signalen, die mit wenig Aufwand empfangen werden können. Das hat zum einen den Grund des deutlich erhöhten Datendurchsatzes in höheren Frequenzen. Zum anderen den technischen Fortschritt in Bezug auf die Dekodierungsverfahren sowie die Modulationsverfahren.  

